% Options for packages loaded elsewhere
\PassOptionsToPackage{unicode}{hyperref}
\PassOptionsToPackage{hyphens}{url}
%
\documentclass[
]{article}
\usepackage{amsmath,amssymb}
\usepackage{iftex}
\ifPDFTeX
  \usepackage[T1]{fontenc}
  \usepackage[utf8]{inputenc}
  \usepackage{textcomp} % provide euro and other symbols
\else % if luatex or xetex
  \usepackage{unicode-math} % this also loads fontspec
  \defaultfontfeatures{Scale=MatchLowercase}
  \defaultfontfeatures[\rmfamily]{Ligatures=TeX,Scale=1}
\fi
\usepackage{lmodern}
\ifPDFTeX\else
  % xetex/luatex font selection
\fi
% Use upquote if available, for straight quotes in verbatim environments
\IfFileExists{upquote.sty}{\usepackage{upquote}}{}
\IfFileExists{microtype.sty}{% use microtype if available
  \usepackage[]{microtype}
  \UseMicrotypeSet[protrusion]{basicmath} % disable protrusion for tt fonts
}{}
\makeatletter
\@ifundefined{KOMAClassName}{% if non-KOMA class
  \IfFileExists{parskip.sty}{%
    \usepackage{parskip}
  }{% else
    \setlength{\parindent}{0pt}
    \setlength{\parskip}{6pt plus 2pt minus 1pt}}
}{% if KOMA class
  \KOMAoptions{parskip=half}}
\makeatother
\usepackage{xcolor}
\usepackage[margin=1in]{geometry}
\usepackage{color}
\usepackage{fancyvrb}
\newcommand{\VerbBar}{|}
\newcommand{\VERB}{\Verb[commandchars=\\\{\}]}
\DefineVerbatimEnvironment{Highlighting}{Verbatim}{commandchars=\\\{\}}
% Add ',fontsize=\small' for more characters per line
\usepackage{framed}
\definecolor{shadecolor}{RGB}{248,248,248}
\newenvironment{Shaded}{\begin{snugshade}}{\end{snugshade}}
\newcommand{\AlertTok}[1]{\textcolor[rgb]{0.94,0.16,0.16}{#1}}
\newcommand{\AnnotationTok}[1]{\textcolor[rgb]{0.56,0.35,0.01}{\textbf{\textit{#1}}}}
\newcommand{\AttributeTok}[1]{\textcolor[rgb]{0.13,0.29,0.53}{#1}}
\newcommand{\BaseNTok}[1]{\textcolor[rgb]{0.00,0.00,0.81}{#1}}
\newcommand{\BuiltInTok}[1]{#1}
\newcommand{\CharTok}[1]{\textcolor[rgb]{0.31,0.60,0.02}{#1}}
\newcommand{\CommentTok}[1]{\textcolor[rgb]{0.56,0.35,0.01}{\textit{#1}}}
\newcommand{\CommentVarTok}[1]{\textcolor[rgb]{0.56,0.35,0.01}{\textbf{\textit{#1}}}}
\newcommand{\ConstantTok}[1]{\textcolor[rgb]{0.56,0.35,0.01}{#1}}
\newcommand{\ControlFlowTok}[1]{\textcolor[rgb]{0.13,0.29,0.53}{\textbf{#1}}}
\newcommand{\DataTypeTok}[1]{\textcolor[rgb]{0.13,0.29,0.53}{#1}}
\newcommand{\DecValTok}[1]{\textcolor[rgb]{0.00,0.00,0.81}{#1}}
\newcommand{\DocumentationTok}[1]{\textcolor[rgb]{0.56,0.35,0.01}{\textbf{\textit{#1}}}}
\newcommand{\ErrorTok}[1]{\textcolor[rgb]{0.64,0.00,0.00}{\textbf{#1}}}
\newcommand{\ExtensionTok}[1]{#1}
\newcommand{\FloatTok}[1]{\textcolor[rgb]{0.00,0.00,0.81}{#1}}
\newcommand{\FunctionTok}[1]{\textcolor[rgb]{0.13,0.29,0.53}{\textbf{#1}}}
\newcommand{\ImportTok}[1]{#1}
\newcommand{\InformationTok}[1]{\textcolor[rgb]{0.56,0.35,0.01}{\textbf{\textit{#1}}}}
\newcommand{\KeywordTok}[1]{\textcolor[rgb]{0.13,0.29,0.53}{\textbf{#1}}}
\newcommand{\NormalTok}[1]{#1}
\newcommand{\OperatorTok}[1]{\textcolor[rgb]{0.81,0.36,0.00}{\textbf{#1}}}
\newcommand{\OtherTok}[1]{\textcolor[rgb]{0.56,0.35,0.01}{#1}}
\newcommand{\PreprocessorTok}[1]{\textcolor[rgb]{0.56,0.35,0.01}{\textit{#1}}}
\newcommand{\RegionMarkerTok}[1]{#1}
\newcommand{\SpecialCharTok}[1]{\textcolor[rgb]{0.81,0.36,0.00}{\textbf{#1}}}
\newcommand{\SpecialStringTok}[1]{\textcolor[rgb]{0.31,0.60,0.02}{#1}}
\newcommand{\StringTok}[1]{\textcolor[rgb]{0.31,0.60,0.02}{#1}}
\newcommand{\VariableTok}[1]{\textcolor[rgb]{0.00,0.00,0.00}{#1}}
\newcommand{\VerbatimStringTok}[1]{\textcolor[rgb]{0.31,0.60,0.02}{#1}}
\newcommand{\WarningTok}[1]{\textcolor[rgb]{0.56,0.35,0.01}{\textbf{\textit{#1}}}}
\usepackage{graphicx}
\makeatletter
\def\maxwidth{\ifdim\Gin@nat@width>\linewidth\linewidth\else\Gin@nat@width\fi}
\def\maxheight{\ifdim\Gin@nat@height>\textheight\textheight\else\Gin@nat@height\fi}
\makeatother
% Scale images if necessary, so that they will not overflow the page
% margins by default, and it is still possible to overwrite the defaults
% using explicit options in \includegraphics[width, height, ...]{}
\setkeys{Gin}{width=\maxwidth,height=\maxheight,keepaspectratio}
% Set default figure placement to htbp
\makeatletter
\def\fps@figure{htbp}
\makeatother
\setlength{\emergencystretch}{3em} % prevent overfull lines
\providecommand{\tightlist}{%
  \setlength{\itemsep}{0pt}\setlength{\parskip}{0pt}}
\setcounter{secnumdepth}{-\maxdimen} % remove section numbering
\ifLuaTeX
  \usepackage{selnolig}  % disable illegal ligatures
\fi
\usepackage{bookmark}
\IfFileExists{xurl.sty}{\usepackage{xurl}}{} % add URL line breaks if available
\urlstyle{same}
\hypersetup{
  pdftitle={Integrative Mendelian randomization for detecting exposure-by-group interactions using group-specific and combined summary statistics},
  pdfauthor={Ke Xu,Nathaniel Maydanchik, Bowei Kang, Jianhai Chen, Qixiang Chen, Gongyao Xu, Shinya Tasaki, David A. Bennett, Lin S. Chen},
  hidelinks,
  pdfcreator={LaTeX via pandoc}}

\title{Integrative Mendelian randomization for detecting
exposure-by-group interactions using group-specific and combined summary
statistics}
\author{Ke Xu,Nathaniel Maydanchik, Bowei Kang, Jianhai Chen, Qixiang
Chen, Gongyao Xu, Shinya Tasaki, David A. Bennett, Lin S. Chen}
\date{}

\begin{document}
\maketitle

\subsubsection{Introduction}\label{introduction}

This vignette provides an introduction to the \texttt{int2MR} package.
The R package \texttt{int2MR} implements the int2MR method for detecting
both direct exposure-outcome effect within comparison group and
reference group and exposure-group interaction effect.

Before installing int2MR, ensure that you have the devtools package
installed. The package also requires rstan for Bayesian modeling. To
install the development version of int2MR, run:

\begin{Shaded}
\begin{Highlighting}[]
\CommentTok{\# Load devtools package}
\FunctionTok{library}\NormalTok{(devtools)}
\end{Highlighting}
\end{Shaded}

\begin{verbatim}
## Loading required package: usethis
\end{verbatim}

\begin{Shaded}
\begin{Highlighting}[]
\CommentTok{\# Install the int2MR package from GitHub}
\FunctionTok{install\_github}\NormalTok{(}\StringTok{"Likeli{-}Ke/int2MR"}\NormalTok{)}
\end{Highlighting}
\end{Shaded}

\begin{verbatim}
## Skipping install of 'int2MR' from a github remote, the SHA1 (998f24b3) has not changed since last install.
##   Use `force = TRUE` to force installation
\end{verbatim}

\begin{Shaded}
\begin{Highlighting}[]
\CommentTok{\# Load the int2MR package}
\FunctionTok{library}\NormalTok{(int2MR)}
\end{Highlighting}
\end{Shaded}

\section{Input Data Format}\label{input-data-format}

The \texttt{int2MR} package supports two types of GWAS summary
statistics:

\begin{itemize}
\tightlist
\item
  \textbf{Three-sample data}: Utilizes three separate IV-to-outcome GWAS
  statistics.
\item
  \textbf{Two-sample data}: Utilizes two IV-to-outcome GWAS summary
  statistics.
\end{itemize}

To detect the exposure-group interaction effect, it is essential that
the proportion of the comparison group (denoted as rho) varies among the
provided GWAS summary statistics.

\subsection{Data Requirements}\label{data-requirements}

Your input data should include:

\begin{itemize}
\tightlist
\item
  \textbf{Number of IVs} (instrumental variables)
\item
  \textbf{Point estimates} and \textbf{squared standard errors} for:

  \begin{itemize}
  \tightlist
  \item
    IV-to-outcome effects
  \item
    IV-to-exposure effects
  \end{itemize}
\item
  \textbf{Proportion of the comparison group (rho)} in each
  IV-to-outcome GWAS summary statistic.
\end{itemize}

Load the example datasets provided with the package as follows:

\begin{Shaded}
\begin{Highlighting}[]
\FunctionTok{data}\NormalTok{(example\_3sample\_data)}
\FunctionTok{data}\NormalTok{(example\_2sample\_data)}
\end{Highlighting}
\end{Shaded}

For further details on the example data formats included with the
package, consult the help pages.

\begin{Shaded}
\begin{Highlighting}[]
\FunctionTok{help}\NormalTok{(example\_3sample\_data)}
\FunctionTok{help}\NormalTok{(example\_2sample\_data)}
\end{Highlighting}
\end{Shaded}

\section{Running Examples}\label{running-examples}

This section demonstrates how to run \texttt{int2MR} using simulated
data. Two examples are provided: one for three-sample data and one for
two-sample data.

\subsection{Example 1: Three-Sample
Data}\label{example-1-three-sample-data}

In this example, we perform an analysis on simulated three-sample data.
The parameters include:

\begin{itemize}
\tightlist
\item
  \textbf{model\_type}: ``3sample''
\item
  \textbf{Prior distributions}: Inverse gamma priors with a shape and
  scale of 0.1
\item
  \textbf{MCMC Settings}: 2 chains, 5000 iterations with a warm-up
  period of 1000 iterations, and an adapt\_delta of 0.95.
\end{itemize}

\begin{Shaded}
\begin{Highlighting}[]
\NormalTok{result\_3sample }\OtherTok{\textless{}{-}} \FunctionTok{int2MR}\NormalTok{(}\AttributeTok{data\_list\_3sample =}\NormalTok{ example\_3sample\_data,}
                 \AttributeTok{model\_type =} \StringTok{"3sample"}\NormalTok{,}
                 \AttributeTok{prior\_inv\_gamma\_shape =} \FloatTok{0.1}\NormalTok{,}
                 \AttributeTok{prior\_inv\_gamma\_scale =} \FloatTok{0.1}\NormalTok{,}
                 \AttributeTok{chains =} \DecValTok{2}\NormalTok{, }\AttributeTok{iter =} \DecValTok{5000}\NormalTok{, }\AttributeTok{warmup =} \DecValTok{1000}\NormalTok{,}
                 \AttributeTok{adapt\_delta =} \FloatTok{0.95}\NormalTok{)}
\end{Highlighting}
\end{Shaded}

\begin{verbatim}
## Loading required package: rstan
\end{verbatim}

\begin{verbatim}
## Loading required package: StanHeaders
\end{verbatim}

\begin{verbatim}
## 
## rstan version 2.32.7 (Stan version 2.32.2)
\end{verbatim}

\begin{verbatim}
## For execution on a local, multicore CPU with excess RAM we recommend calling
## options(mc.cores = parallel::detectCores()).
## To avoid recompilation of unchanged Stan programs, we recommend calling
## rstan_options(auto_write = TRUE)
## For within-chain threading using `reduce_sum()` or `map_rect()` Stan functions,
## change `threads_per_chain` option:
## rstan_options(threads_per_chain = 1)
\end{verbatim}

\begin{verbatim}
## Trying to compile a simple C file
\end{verbatim}

\begin{verbatim}
## Running /Library/Frameworks/R.framework/Resources/bin/R CMD SHLIB foo.c
## using C compiler: ‘Apple clang version 16.0.0 (clang-1600.0.26.6)’
## using SDK: ‘’
## clang -arch arm64 -I"/Library/Frameworks/R.framework/Resources/include" -DNDEBUG   -I"/Library/Frameworks/R.framework/Versions/4.4-arm64/Resources/library/Rcpp/include/"  -I"/Library/Frameworks/R.framework/Versions/4.4-arm64/Resources/library/RcppEigen/include/"  -I"/Library/Frameworks/R.framework/Versions/4.4-arm64/Resources/library/RcppEigen/include/unsupported"  -I"/Library/Frameworks/R.framework/Versions/4.4-arm64/Resources/library/BH/include" -I"/Library/Frameworks/R.framework/Versions/4.4-arm64/Resources/library/StanHeaders/include/src/"  -I"/Library/Frameworks/R.framework/Versions/4.4-arm64/Resources/library/StanHeaders/include/"  -I"/Library/Frameworks/R.framework/Versions/4.4-arm64/Resources/library/RcppParallel/include/"  -I"/Library/Frameworks/R.framework/Versions/4.4-arm64/Resources/library/rstan/include" -DEIGEN_NO_DEBUG  -DBOOST_DISABLE_ASSERTS  -DBOOST_PENDING_INTEGER_LOG2_HPP  -DSTAN_THREADS  -DUSE_STANC3 -DSTRICT_R_HEADERS  -DBOOST_PHOENIX_NO_VARIADIC_EXPRESSION  -D_HAS_AUTO_PTR_ETC=0  -include '/Library/Frameworks/R.framework/Versions/4.4-arm64/Resources/library/StanHeaders/include/stan/math/prim/fun/Eigen.hpp'  -D_REENTRANT -DRCPP_PARALLEL_USE_TBB=1   -I/opt/R/arm64/include    -fPIC  -falign-functions=64 -Wall -g -O2  -c foo.c -o foo.o
## In file included from <built-in>:1:
## In file included from /Library/Frameworks/R.framework/Versions/4.4-arm64/Resources/library/StanHeaders/include/stan/math/prim/fun/Eigen.hpp:22:
## In file included from /Library/Frameworks/R.framework/Versions/4.4-arm64/Resources/library/RcppEigen/include/Eigen/Dense:1:
## In file included from /Library/Frameworks/R.framework/Versions/4.4-arm64/Resources/library/RcppEigen/include/Eigen/Core:19:
## /Library/Frameworks/R.framework/Versions/4.4-arm64/Resources/library/RcppEigen/include/Eigen/src/Core/util/Macros.h:679:10: fatal error: 'cmath' file not found
##   679 | #include <cmath>
##       |          ^~~~~~~
## 1 error generated.
## make: *** [foo.o] Error 1
\end{verbatim}

\begin{Shaded}
\begin{Highlighting}[]
\CommentTok{\# Display the results for the three{-}sample analysis}
\NormalTok{result\_3sample}\SpecialCharTok{$}\NormalTok{result\_3sample}
\end{Highlighting}
\end{Shaded}

\begin{verbatim}
##    est_beta    se_beta   pval_beta est_beta_int se_beta_int pval_beta_int
## 1 0.1079651 0.03820672 0.004716041   -0.0629725  0.07122689     0.3766361
##   total_effect pval_total
## 1   0.04499265  0.3044504
\end{verbatim}

\subsection{Example 2: Two-Sample Data}\label{example-2-two-sample-data}

This example demonstrates an analysis using simulated two-sample data.
The configuration is similar to the three-sample analysis, with
\texttt{model\_type} set to ``2sample''.

\begin{Shaded}
\begin{Highlighting}[]
\NormalTok{result\_2sample }\OtherTok{\textless{}{-}} \FunctionTok{int2MR}\NormalTok{(}\AttributeTok{data\_list\_2sample =}\NormalTok{ example\_2sample\_data,}
                 \AttributeTok{model\_type =} \StringTok{"2sample"}\NormalTok{,}
                 \AttributeTok{prior\_inv\_gamma\_shape =} \FloatTok{0.1}\NormalTok{,}
                 \AttributeTok{prior\_inv\_gamma\_scale =} \FloatTok{0.1}\NormalTok{,}
                 \AttributeTok{chains =} \DecValTok{2}\NormalTok{, }\AttributeTok{iter =} \DecValTok{5000}\NormalTok{, }\AttributeTok{warmup =} \DecValTok{1000}\NormalTok{,}
                 \AttributeTok{adapt\_delta =} \FloatTok{0.95}\NormalTok{)}
\end{Highlighting}
\end{Shaded}

\begin{verbatim}
## Trying to compile a simple C file
\end{verbatim}

\begin{verbatim}
## Running /Library/Frameworks/R.framework/Resources/bin/R CMD SHLIB foo.c
## using C compiler: ‘Apple clang version 16.0.0 (clang-1600.0.26.6)’
## using SDK: ‘’
## clang -arch arm64 -I"/Library/Frameworks/R.framework/Resources/include" -DNDEBUG   -I"/Library/Frameworks/R.framework/Versions/4.4-arm64/Resources/library/Rcpp/include/"  -I"/Library/Frameworks/R.framework/Versions/4.4-arm64/Resources/library/RcppEigen/include/"  -I"/Library/Frameworks/R.framework/Versions/4.4-arm64/Resources/library/RcppEigen/include/unsupported"  -I"/Library/Frameworks/R.framework/Versions/4.4-arm64/Resources/library/BH/include" -I"/Library/Frameworks/R.framework/Versions/4.4-arm64/Resources/library/StanHeaders/include/src/"  -I"/Library/Frameworks/R.framework/Versions/4.4-arm64/Resources/library/StanHeaders/include/"  -I"/Library/Frameworks/R.framework/Versions/4.4-arm64/Resources/library/RcppParallel/include/"  -I"/Library/Frameworks/R.framework/Versions/4.4-arm64/Resources/library/rstan/include" -DEIGEN_NO_DEBUG  -DBOOST_DISABLE_ASSERTS  -DBOOST_PENDING_INTEGER_LOG2_HPP  -DSTAN_THREADS  -DUSE_STANC3 -DSTRICT_R_HEADERS  -DBOOST_PHOENIX_NO_VARIADIC_EXPRESSION  -D_HAS_AUTO_PTR_ETC=0  -include '/Library/Frameworks/R.framework/Versions/4.4-arm64/Resources/library/StanHeaders/include/stan/math/prim/fun/Eigen.hpp'  -D_REENTRANT -DRCPP_PARALLEL_USE_TBB=1   -I/opt/R/arm64/include    -fPIC  -falign-functions=64 -Wall -g -O2  -c foo.c -o foo.o
## In file included from <built-in>:1:
## In file included from /Library/Frameworks/R.framework/Versions/4.4-arm64/Resources/library/StanHeaders/include/stan/math/prim/fun/Eigen.hpp:22:
## In file included from /Library/Frameworks/R.framework/Versions/4.4-arm64/Resources/library/RcppEigen/include/Eigen/Dense:1:
## In file included from /Library/Frameworks/R.framework/Versions/4.4-arm64/Resources/library/RcppEigen/include/Eigen/Core:19:
## /Library/Frameworks/R.framework/Versions/4.4-arm64/Resources/library/RcppEigen/include/Eigen/src/Core/util/Macros.h:679:10: fatal error: 'cmath' file not found
##   679 | #include <cmath>
##       |          ^~~~~~~
## 1 error generated.
## make: *** [foo.o] Error 1
\end{verbatim}

\begin{Shaded}
\begin{Highlighting}[]
\CommentTok{\# Display the results for the two{-}sample analysis}
\NormalTok{result\_2sample}\SpecialCharTok{$}\NormalTok{result\_2sample}
\end{Highlighting}
\end{Shaded}

\begin{verbatim}
##     est_beta    se_beta  pval_beta est_beta_int se_beta_int pval_beta_int
## 1 0.09833315 0.04382994 0.02486359  -0.07116637  0.07233457     0.3251893
##   total_effect pval_total
## 1   0.02716678  0.6420072
\end{verbatim}

\end{document}
